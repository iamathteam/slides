\documentclass[aspectratio=169,xcolor=dvipsnames]{beamer}
\makeatletter
\def\input@path{{theme/}}
\makeatother
\usetheme{CleanEasy}
\usepackage[utf8]{inputenc}
\usepackage{lmodern}
\usepackage[T1]{fontenc}
\usepackage{fix-cm}
\usepackage{amsmath}
\usepackage{mathtools}
\usepackage{listings}
\usepackage{xcolor}
\usepackage{hyperref}
\usepackage{graphicx}
\usepackage{booktabs}
\usepackage{tikz}
\usetikzlibrary{positioning, shapes, arrows, calc, decorations.pathreplacing, arrows.meta, backgrounds, patterns, overlay-beamer-styles, angles, quotes, intersections}
\usepackage{etoolbox}
\usepackage{animate}

\lstset{
  basicstyle=\ttfamily\small,
  keywordstyle=\color{blue},
  commentstyle=\color{green!60!black},
  stringstyle=\color{red},
  showstringspaces=false,
  breaklines=true,
  frame=single,
  rulecolor=\color{black!30},
  backgroundcolor=\color{black!5},
  numbers=left,
  numberstyle=\tiny\color{black!70},
  numbersep=5pt
}

%----------------------------------------------------------------------------------------
%    TITLE PAGE
%----------------------------------------------------------------------------------------

\title[Modular Arithmetic]{Modular Arithmetic}

\author[IA Math Team]{IA Math Team}

\vspace{-2cm}\date{Feb. 5, 2026}
%---------------------------------------------------------------------------------------

\begin{document}

\begin{frame}[t]
  \titlepage
\end{frame}
\section{Introduction}
\begin{frame}[t]{Competitions}
We have some upcoming competitions. Please sign up!!
\vspace{1cm}
\begin{enumerate}
    \item Floyd County - 2/21
    \item Westminster Math Competition - 3/14
    % \item GCTM Varsity State 4/18
\end{enumerate}

\end{frame}

\begin{frame}[t]{T-Shirts}

\begin{figure}[ht]
    \centering
    \begin{minipage}{0.45\linewidth}
        \centering
        \includegraphics[width=0.8\linewidth]{frontnew.png}
        \label{new}
    \end{minipage}\hfill
    \begin{minipage}{0.45\linewidth}
        \centering
        \includegraphics[width=0.8\linewidth]{backnew.png}
        \label{back}
    \end{minipage}
\end{figure}
IF YOU WOULD LIKE A T-SHIRT AND HAVE ALREADY PAID DUES, PLEASE SUBMIT THE FORM RIGHT NOW.
\end{frame}

\section{Basics}
\begin{frame}{What is Mod?}
\begin{definition}
    For two integers $m$ and $n$, define $m$ \textit{modulo} $n$ as the remainder when $m$ is divided by $n$. We denote this by $m\pmod{n}$.
\end{definition}
\begin{definition}
    We say that $a$ is congruent to $b$ modulo $n$ when $a$ and $b$ have the same remainder when divided by $n$, or equivalently $a-b$ is divisible by $n$. This is denoted
    \[a\equiv b\pmod n\]
\end{definition}
    
\end{frame}
\begin{frame}{Blackbird Quiz}
    1. Find $15 \pmod{4}$ \\
    \vspace{1cm}
    2. Find $18 \pmod{5}$ \\
    \vspace{1cm}
    3. Find a number $a\neq12$ that is \[a\equiv 12\pmod{16}\]
\end{frame}
\begin{frame}{Negative Mods}
        Some of these terms can also be negative and this may be very helpful while solving some problems. \\
    For example: $-1 \equiv 11\pmod{12}$ \\
    For example: $-1 \equiv 11\pmod{6}$ \\
    For example: $-1 \equiv 7\pmod{8}$ 
\end{frame}

\begin{frame}{F1 Quiz}
1. Find 3 negative values that satisfies $X$. $X\equiv 9\pmod{14}$    
\end{frame}


\begin{frame}{Properties of Mods}
    \begin{definition}
        The modulo function can also be added and multiplied quite easily.\\
        $a\pmod{n}+b\pmod{n}  = a+b\pmod{n}$\\
        $a\pmod{n}*b\pmod{n}  = a*b\pmod{n}$\\
    \end{definition}
\end{frame}

\begin{frame}{Example for Addition}
    $a\pmod{n}+b\pmod{n}  = a+b\pmod{n}$\\
    \vspace{1cm}
For example if we want to find $21+67\pmod{10}$ \\
\vspace{1cm}
This will be the same as $21\pmod{10}+67\pmod{10}$
\end{frame}

\begin{frame}{Expeditious Quiz}
    1. What is $34+71\pmod{11}$
\end{frame}

\begin{frame}{Example for Multiplication}
    $a\pmod{n}*b\pmod{n}  = a*b\pmod{n}$\\
    \vspace{1cm}
For example if we want to find $13*16\pmod{10}$ \\
\vspace{1cm}
This will be the same as $13\pmod{10}*16\pmod{10}$
\end{frame}

\begin{frame}{Cheetah Quiz}
    1. What is $51*84\pmod{13}$
\end{frame}

\begin{frame}{Exponents}
    We have that if $a\equiv b\pmod{n}$ then for any integer $m$, \[a^m\equiv b^m\pmod n\], by using the multiplication property.
\end{frame}
\begin{frame}{Swift Quiz}
    Find the remainder when $30^{2026}$ is divided by $29$.
\end{frame}
\begin{frame}{Chinese Remainder Theorem}
    \begin{theorem}
    For ANY integers $k,m,n$ with $m$ and $n$ relatively prime, $k \pmod{mn}$ corresponds with two values $k\pmod n$ and $k\pmod m$ and vice versa. 
    \end{theorem}
    For example note that $6=2\cdot3$. If we want to find, for example, $101\pmod6$, then we could find $101\pmod2$ and $101\pmod 3$, which are $1$ and $2$, respectively. Now we know that $1\pmod2$ and $2\pmod3$ must satisfy $5\pmod6$. We see how this works in an example.
\end{frame}

\begin{frame}{Rapid Quiz}
    Find the remainder when $22^{22}$ is divided by $14$. (Hint: $14=2\cdot7$.)
\end{frame}
\end{document}