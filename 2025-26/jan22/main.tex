\documentclass[aspectratio=169,xcolor=dvipsnames]{beamer}
\makeatletter
\def\input@path{{theme/}}
\makeatother
\usetheme{CleanEasy}
\usepackage[utf8]{inputenc}
\usepackage{lmodern}
\usepackage[T1]{fontenc}
\usepackage{fix-cm}
\usepackage{amsmath}
\usepackage{mathtools}
\usepackage{listings}
\usepackage{xcolor}
\usepackage{hyperref}
\usepackage{graphicx}
\usepackage{booktabs}
\usepackage{tikz}
\usetikzlibrary{positioning, shapes, arrows, calc, decorations.pathreplacing, arrows.meta, backgrounds, patterns, overlay-beamer-styles, angles, quotes, intersections}
\usepackage{etoolbox}
\usepackage{animate}

\lstset{
  basicstyle=\ttfamily\small,
  keywordstyle=\color{blue},
  commentstyle=\color{green!60!black},
  stringstyle=\color{red},
  showstringspaces=false,
  breaklines=true,
  frame=single,
  rulecolor=\color{black!30},
  backgroundcolor=\color{black!5},
  numbers=left,
  numberstyle=\tiny\color{black!70},
  numbersep=5pt
}

%----------------------------------------------------------------------------------------
%    TITLE PAGE
%----------------------------------------------------------------------------------------

\title[Number Theory]{GCD, LCM, and Euclidean Algorithm}

\author[IA Math Team]{IA Math Team}

\vspace{-2cm}\date{Jan. 22, 2026}
%---------------------------------------------------------------------------------------

\begin{document}

\begin{frame}[t]
  \titlepage
\end{frame}

\begin{frame}[t]{Contents}
  \tableofcontents
\end{frame}

\section{Simple Start}

\begin{frame}{GCD \& LCM}
    \begin{definition}
        GCD: greatest common divisor of two numbers
        
        LCM: least common multiple of two numbers.
    \end{definition}
    \begin{exampleblock}{Note}The GCD of two numbers is 1 if and only if the numbers are relatively prime.
    \end{exampleblock}

    \vspace{.5cm}
    GCD of $12$ and $21$ $\rightarrow$
    
    \break
    
    LCM of $8$ and $14$ $\rightarrow$
\end{frame}

\begin{frame}[t]{Quick Quiz}
    Question 1: Find $\gcd(134, 335)$.
    \vspace{3cm}
    
    Question 2: Find the least common multiple of the same numbers above.
\end{frame}
\begin{frame}[t]{More GCD and LCM}
    \begin{theorem}
        For any two numbers $a$ and $b$, the product of their greatest common divisor and least common multiple will always be equal to the product of the two numbers.
        $$\mathrm{lcm}(a, b)\cdot \gcd(a,b)=ab$$
    \end{theorem}
\vspace{2cm}
    Try a few examples!
\end{frame}

\begin{frame}[t]{Quicker Quiz}
    \begin{enumerate}
        \item 
    \end{enumerate}
\end{frame}

\section{Euclidean Algorithm}

\begin{frame}[t]{How do you find GCDs?}
    \begin{itemize}
        \item $\gcd(134, 335)$
        \item $\gcd(2026, 677)$
        \item $\gcd(2267, 20677)$
        \item $\vdots$
    \end{itemize}
\end{frame}

\begin{frame}[t]{So Euclid wanted to find $\gcd(2267,20677)$...}
    \begin{theorem}[Euclidean Algorithm]
        For integers $a,b,c$ with $b\neq0$:
        {\Huge$$ \gcd(a,b)=\gcd(a-bc,b)$$}
        \vspace{0.3cm}
    \end{theorem}
    
\end{frame}
\begin{frame}[t]{So Euclid wanted to find $\gcd(2267,20677)$...}
    $\gcd(2267,20677)=$
    
\end{frame}
\begin{frame}[t]{So Euclid wanted to find $\gcd(2267,20677)$...}
    \begin{align*}&=\gcd({\color{cyan}2267},{\color{red}20677})\\&=\gcd({\color{cyan}2267},{\color{red}20677}-{\color{cyan}2267}\cdot{\color{purple}9})\\
    &=\gcd(2267,274)\\
    \\
    &=\gcd({\color{red}2267},{\color{cyan}274})\\
    &=\gcd({\color{red}2267}-{\color{cyan}274}\cdot{\color{purple}8},{\color{cyan}274})\\
    &=\gcd(105,274)\\
    &=1
    \end{align*}
    
\end{frame}

\section{Worksheet Time!}


\end{document}
