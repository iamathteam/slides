%----------------------------------------------------------------------------------------
%    PACKAGES AND THEMES
%----------------------------------------------------------------------------------------
\documentclass[aspectratio=169,xcolor=dvipsnames]{beamer}
\makeatletter
\def\input@path{{theme/}}
\makeatother
\usetheme{CleanEasy}
\usepackage[utf8]{inputenc}
\usepackage{lmodern}
\usepackage[T1]{fontenc}
% \usepackage[brazil]{babel}
\usepackage{fix-cm}
\usepackage{amsmath}
\usepackage{mathtools}
\usepackage{listings}
\usepackage{xcolor}
\usepackage{hyperref}
\usepackage{graphicx} % Allows including images
\usepackage{booktabs} % Allows the use of \toprule, \midrule and \bottomrule in tables
\usepackage{tikz}
\usetikzlibrary{positioning, shapes, arrows, calc, decorations.pathreplacing, arrows.meta, backgrounds, patterns, overlay-beamer-styles}
\usepackage{etoolbox}
\usepackage{animate}

%----------------------------------------------------------------------------------------
%    LAYOUT CONFIGURATION
%----------------------------------------------------------------------------------------
\input{configs/configs}
%----------------------------------------------------------------------------------------
%    TITLE PAGE
%----------------------------------------------------------------------------------------
\input{configs/title_page} 
%----------------------------------------------------------------------------------------


\begin{document}

\begin{frame}[t]
  \titlepage
\end{frame}

\begin{frame}[t]{Contents}
  \tableofcontents
\end{frame}

\section{Polynomials}

\begin{frame}[t]{What is a polynomial?}
    \begin{definition}
        A polynomial is a function $f(x)$ written in terms of a sum of nonnegative integer powers. It can be written as
        \[f(x)=a_nx^n+a_{n-1}x^{n-1}+\cdots +a_1x+a_0\]
        for $n\ge0$.
    
    \end{definition}
    
    \vspace{0.5cm}
    
    The following function would be considered a polynomial:
    \[
    g(x)=5x^2-3x+1
    \]
\end{frame}

\begin{frame}[t]{Roots}
    Polynomials have interesting properties, such as their roots.
    
    \begin{definition}
        A root $r$ of a polynomial $f(x)$ is a number such that $f(r)=0$.
    \end{definition}
    
    Find the roots of the following polynomials:
    \[f(x)=x+1\]
    \[g(x)=x^2-2x+1\]
    \[h(x)=x^2+x+67\]
\end{frame}

\begin{frame}[t]{Finding Roots}
    A common method for finding roots is \textbf{factoring}. That is, some polynomials can be written as a product of terms involving their roots.
    \[
    x^2-3x+2=(x-1)(x-2)
    \]
    In fact, the following fact holds.

    \begin{theorem}[Fundamental Theorem of Algebra]
        All polynomials can be expressed as a product of monomials with their roots and the leading coefficient.
        \[f(x)=a(x-r_1)(x-r_2)\cdots(x-r_n)\]
    \end{theorem}

    \vspace{0.5cm}

    Note that \(r_i\) do not need to distinct (and may also be complex!)
\end{frame}

\section{Vieta's Formulae}

\begin{frame}[t]{Vieta's Formulae}
    Take the following
    \[
    f(x)=x^2-11x+18
    \]
    
    \begin{itemize}
        \item Factor the polynomial.
        \item Find the roots.
        \item What do you notice about the coefficients and the roots?
    \end{itemize}
\end{frame}

\begin{frame}[t]{Quadratics}
    Now, look at the general quadratic case.
    \[
    f(x)=a(x-r_1)(x-r_2)=ax^2-a(r_1+r_2)x+ar_1r_2
    \]
    
    \begin{theorem}[Vieta's Formula for Quadratics]
        When given a quadratic of the form
        \[f(x)=ax^2+bx+c,\]
        the sum of the roots is $-\frac ba$, and the product is $\frac ca$.
    \end{theorem}
\end{frame}

\begin{frame}[t]{Practice}
    \begin{itemize}
        \item If the following \(f(x)\) has roots \(\alpha, \beta\), find \(\alpha+\beta\) and \(\alpha\beta\).
        \[f(x)=3x^2-18x+24\]
        \item A rectangular garden has a fixed area of \(30 \rm m^2\). Also, the length of the garden is \(5 \rm m\) more than its width. 
    \end{itemize}
\end{frame}

\begin{frame}[t]{The General Form of Vieta's}
    It turns out that this formula works in general for any polynomial. First, we need to define some terms:
    \begin{definition}[Elementary Symmetric Polynomials]
        The elementary symmetric polynomial $e_i$ is defined as
        \[\sum_{1\le a_1<a_2<\cdots<a_i\le n}r_{a_1}r_{a_2}\cdots r_{a_i}\]     
        
    \end{definition}
\end{frame}

\begin{frame}[t]{The General Form of Vieta's (cont.)}
    For example, we have
    \[
        \begin{aligned}
        e_1 &= r_1+r_2+\cdots+r_{n-1}+r_n\\
        e_2 &=r_1r_2+r_1r_3+\cdots +r_1r_n\cdots +r_{n-1}r_n= \sum_{1\le i<j\le n} r_i r_j\\
        &\hspace{0.2cm} \vdots\\
        e_k &= \sum_{1\le i_1<\cdots<i_k\le n} r_{i_1}\cdots r_{i_k}
        \end{aligned}
        \]
\end{frame}
\begin{frame}[t]{The General Form of Vieta's (cont. cont.)}
    Finally, we arrive at the result
    \begin{theorem}[Vieta's formulae]
        In a polynomial \[f(x)=a_nx^n+a_{n-1}x^{n-1}\cdots a_1x+a_0\] with roots $r_1,r_2,\cdots,r_n$, with elementary symmetric polynomials $e_1,e_2,\cdots,e_n$, we have that
        \[e_i=\frac{(-1)^i\cdot a_{n-i}}{a_n}\]
    \end{theorem}
\end{frame}
\section{Problems}
\begin{frame}[t]{Problems}
    \begin{enumerate}
        \item Let \(r_1, r_2\) be the roots of \(f(x)=x^2-5x+4\). Find \((r_1-r_2)^2\).
        \item \textit{Challenge Problem: } Let the function 
        \[g(x) = x^4 - 5x^3 + 2x^2 +7x - 11,\] and let the roots of $g(x)$ be $p,q,r,s$. Find the following: \\
        \begin{enumerate}[a)]
            \item $pqr+qrs+prs+pqs$
            \item $\dfrac{1}{p}+\dfrac{1}{q}+\dfrac{1}{r}+\dfrac{1}{s}$
            \item $p^2 +q^2 + r^2 + s^2$
            \item $p^2qrs + pq^2rs + pqr^2s + pqrs^2$
        \end{enumerate} 
    \end{enumerate}
\end{frame}

% \begin{frame}[t]{Lists and Numbering}
%   \begin{columns}
%     \begin{column}{0.48\textwidth}
%       Bulleted list:
%       \begin{itemize}
%         \item First level item
%         \begin{itemize}
%           \item Second level item
%           \item Another second level
%           \begin{itemize}
%             \item Third level item
%           \end{itemize}
%         \end{itemize}
%         \item Another first level item
%       \end{itemize}
%     \end{column}
%     \begin{column}{0.48\textwidth}
%       Numbered list:
%       \begin{enumerate}
%         \item First step
%         \begin{enumerate}
%           \item Substep one
%           \item Substep two
%         \end{enumerate}
%         \item Second step
%         \item Third step
%       \end{enumerate}
%     \end{column}
%   \end{columns}
% \end{frame}

% \begin{frame}[t]{Tables}
%   \begin{table}
%     \centering
%     \caption{Sample table with booktabs style}
%     \begin{tabular}{lcr}
%       \toprule
%       \textbf{Header 1} & \textbf{Header 2} & \textbf{Header 3} \\
%       \midrule
%       Row 1, Col 1 & Row 1, Col 2 & 123.45 \\
%       Row 2, Col 1 & Row 2, Col 2 & 67.89 \\
%       Row 3, Col 1 & Row 3, Col 2 & 456.78 \\
%       \bottomrule
%     \end{tabular}
%   \end{table}
  
%   \vspace{0.5cm}
  
%   \begin{block}{Table styling}
%     The CleanEasy theme works well with the booktabs package for professional-looking tables. 
%     Simple color alterations make tables more readable without being distracting.
%   \end{block}
% \end{frame}

% \begin{frame}[t]{Mathematical Equations}
%   The CleanEasy theme includes proper mathematical typesetting:
  
%   \begin{align}
%     E &= mc^2 \\
%     F &= G\frac{m_1 m_2}{r^2}
%   \end{align}
  
%   Maxwell's equations in differential form:
%   \begin{align}
%     \nabla \cdot \vec{E} &= \frac{\rho}{\varepsilon_0} \\
%     \nabla \cdot \vec{B} &= 0 \\
%     \nabla \times \vec{E} &= -\frac{\partial \vec{B}}{\partial t} \\
%     \nabla \times \vec{B} &= \mu_0 \vec{J} + \mu_0\varepsilon_0\frac{\partial \vec{E}}{\partial t}
%   \end{align}
  
%   Inline equations like $E = mc^2$ are also properly rendered.
% \end{frame}

% \begin{frame}[fragile]{Code Listings}
%   \begin{lstlisting}[language=Python]
% # A simple Python function
% def fibonacci(n):
%     """Return the nth Fibonacci number"""
%     if n <= 0:
%         return 0
%     elif n == 1:
%         return 1
%     else:
%         a, b = 0, 1
%         for _ in range(2, n + 1):
%             a, b = b, a + b
%         return b
  
% # Calculate the 10th Fibonacci number
% result = fibonacci(10)
% print(f"The 10th Fibonacci number is {result}")
%   \end{lstlisting}
% \end{frame}

% \begin{frame}[t]{Figures and Graphics}
%   \begin{columns}
%     \begin{column}{0.48\textwidth}
%       \centering
%       \begin{figure}
%         \centering
%         % Replace with actual image path
%         % \includegraphics[width=\textwidth]{figures/sample_image.png}
%         \tikz \fill[blue!30] (0,0) rectangle (3,2);
%         \caption{Sample placeholder image}
%       \end{figure}
%     \end{column}
%     \begin{column}{0.48\textwidth}
%       \centering
%       \begin{figure}          
%       \begin{tikzpicture}
%         % Simple TikZ diagram
%         \draw[thick, ->] (0,0) -- (4,0) node[right] {$x$};
%         \draw[thick, ->] (0,0) -- (0,3) node[above] {$y$};
%         \draw[blue, thick] (0,0) .. controls (1,3) and (3,1) .. (4,2);
%         \node at (2,2.5) {$f(x)$};
%       \end{tikzpicture}
%       \caption{Simple TikZ diagram}
%        \label{fig:enter-label}
%       \end{figure}
%     \end{column}
%   \end{columns}
% \end{frame}

% \section{Results}
% \begin{frame}[t]{Overlays and Animations}
%   Beamer supports step-by-step revelations:
  
%   \begin{itemize}
%     \item<1-> First point appears on slide 1
%     \item<2-> Second point appears on slide 2
%     \item<3-> Third point appears on slide 3
%   \end{itemize}
  
%   \pause
  
%   This text appears after a pause.
  
%   \onslide<4->{
%     And this content appears on slide 4.
%   }
  
%   \begin{block}<5->{Delayed Block}
%     This entire block appears only on slide 5.
%   \end{block}
% \end{frame}

% \begin{frame}[t]{Citations and References}
%   CleanEasy works well with bibliographies and citations:
  
%   \begin{block}{Sample citation}
%     According to Einstein \cite{einstein1905}, space and time are relative.
%   \end{block}
  
%   \begin{exampleblock}{Bibliography management}
%     The theme is compatible with BibTeX, BibLaTeX, and other bibliography management tools.
%   \end{exampleblock}
  
%   % Sample bibliography entries (not functional without .bib file)
%   % \begin{thebibliography}{9}
%   % \bibitem{einstein1905}
%   %   Albert Einstein.
%   %   \emph{On the Electrodynamics of Moving Bodies}.
%   %   Annalen der Physik, 1905.
%   % \end{thebibliography}
% \end{frame}

% \begin{frame}[t]{Custom TikZ Graphics}

%   \begin{figure}
%       \centering
%       \begin{tikzpicture}[scale=0.7]
%         % Coordinate axes
%         \draw[thick, ->] (-3,0) -- (3,0) node[right] {$x$};
%         \draw[thick, ->] (0,-3) -- (0,3) node[above] {$y$};
        
%         % Unit circle
%         \draw[blue!80, thick] (0,0) circle (2);
        
%         % Angle and point on circle
%         \filldraw[red] (60:2) circle (2pt);
%         \draw[red, thick] (0,0) -- (60:2);
%         \draw[red, thick, ->] (0.5,0) arc (0:60:0.5);
%         \node[red] at (30:0.7) {$\theta$};
        
%         % Coordinates
%         \draw[dashed] (60:2) -- (60:2 |- 0,0) node[below] {$\cos\theta$};
%         \draw[dashed] (60:2) -- (0,0 -| 60:2) node[left] {$\sin\theta$};
%       \end{tikzpicture}
%       \caption{The unit circle with trigonometric functions}
%       \label{fig:exampleTikz}
%   \end{figure}
% \end{frame}

% \section{Conclusions}
% \begin{frame}[t]{Theme Customization}
%   The CleanEasy theme can be easily customized:
  
%   \begin{itemize}
%     \item Edit \texttt{beamercolorthemeCleanEasy.sty} to change colors
%     \item Modify \texttt{beamerfontthemeCleanEasy.sty} for different fonts
%     \item Adjust \texttt{beamerinnerthemeCleanEasy.sty} for layout changes
%     \item Update \texttt{configs.tex} for footer and section page customization
%   \end{itemize}
  
%   \begin{alertblock}{Important Note}
%     Always maintain consistent design elements throughout your presentation for a professional look.
%   \end{alertblock}
% \end{frame}

% \begin{frame}[t]{Final Thoughts}
%   \begin{block}{Benefits of CleanEasy}
%     \begin{itemize}
%       \item Professional appearance suitable for academic and business contexts
%       \item Careful attention to typography and spacing
%       \item High readability with suitable contrast ratios
%       \item Flexible design that works with different content types
%     \end{itemize}
%   \end{block}
  
%   \vspace{0.5cm}
  
%   \begin{center}
%     \large{The CleanEasy theme is designed to let your content shine without distractions}
%   \end{center}
% \end{frame}

% \begin{frame}[t]
%   \centering
%   \Huge \textbf{Thank you!}
  
%   \vspace{1cm}
%   \normalsize
%   \href{mailto:your@email.com}{your@email.com}
  
%   \vspace{0.5cm}
%   \small
%   \texttt{https://someurl.com}
% \end{frame}

% Sample bibliography (not shown in presentation, just for reference)
% \begin{frame}[t]{References}
%   \begin{thebibliography}{9}
%     \bibitem{einstein1905}
%       Albert Einstein.
%       \emph{On the Electrodynamics of Moving Bodies}.
%       Annalen der Physik, 1905.
%    
%     \bibitem{beamer}
%       Till Tantau.
%       \emph{The Beamer Class}.
%       \url{https://ctan.org/pkg/beamer}
%   \end{thebibliography}
% \end{frame}

% \begin{frame}[t]{References}
%   \bibliography{reference.bib}
%   \bibliographystyle{apalike}
% \end{frame}

\end{document}