\documentclass[aspectratio=169,xcolor=dvipsnames]{beamer}
\makeatletter
\def\input@path{{theme/}}
\makeatother
\usetheme{CleanEasy}
\usepackage[utf8]{inputenc}
\usepackage{lmodern}
\usepackage[T1]{fontenc}
\usepackage{fix-cm}
\usepackage{amsmath}
\usepackage{mathtools}
\usepackage{listings}
\usepackage{xcolor}
\usepackage{hyperref}
\usepackage{graphicx}
\usepackage{booktabs}
\usepackage{tikz}
\usetikzlibrary{positioning, shapes, arrows, calc, decorations.pathreplacing, arrows.meta, backgrounds, patterns, overlay-beamer-styles}
\usepackage{etoolbox}
\usepackage{animate}
\lstset{
  basicstyle=\ttfamily\small,
  keywordstyle=\color{blue},
  commentstyle=\color{green!60!black},
  stringstyle=\color{red},
  showstringspaces=false,
  breaklines=true,
  frame=single,
  rulecolor=\color{black!30},
  backgroundcolor=\color{black!5},
  numbers=left,
  numberstyle=\tiny\color{black!70},
  numbersep=5pt
}
%----------------------------------------------------------------------------------------
%    TITLE PAGE
%----------------------------------------------------------------------------------------

\title[Intermediate Combinatorics I]{Intermediate Combinatorics I}

\author[IA Math Team]{IA Math Team}

\vspace{-2cm}\date{Oct. 9, 2025}
%---------------------------------------------------------------------------------------


\begin{document}

\begin{frame}[t]
  \titlepage
\end{frame}

\begin{frame}[t]{Contents}
  \tableofcontents
\end{frame}


\section{Reminders}

\begin{frame}[t]{Upcoming Competitions and Events}
    Please \textit{\textbf{sign up}}!
    \vspace{0.2cm}
    \begin{enumerate}
        \item Georgia Math League \# 1  % gauge how many papers we need to print out
        \begin{itemize}
            \item October 23!
        \end{itemize}
        \item UGA
        \begin{itemize}
            \item October 25!
            \item In-person; Zell B. Miller Learning Center (right across nice dining hall)
        \end{itemize}
        \item KSU
        \begin{itemize}
            \item October 29!
            \item At school
        \end{itemize}
        \item AMC 10/12
        \begin{itemize}
            \item A's on November 5, B's on November 13
        \end{itemize}
    \end{enumerate}
\end{frame}

\section{Stars and Bars}

\begin{frame}[t]{Types}
    Recap of problem types from last practice:
    \begin{itemize}
        \item Ordering distinct items
        \item Picking distinct items with and without order
        \item Rearranging strings (grid movement)
        \item Arranging people in a circle
    \end{itemize}
    
    \vspace{2.067cm}
    
    This week we will be focusing on the problems with indistinguishable items.
\end{frame}

\begin{frame}[t]{A classic problem}
    If we have 9 candies, how many ways are there for us to distribute them to 3 different people.

    \begin{itemize}
        \item How do we count these?
    \end{itemize}
    
    \vspace{4cm}
    
    We will need to use a method called stars and bars to solve this question.
\end{frame}

\begin{frame}[t]{Framing Stars and Bars}
    Consider the following setup
    \[
        \star\hspace{0.1cm}
        \star\hspace{0.1cm}
        \star\hspace{0.1cm}
        \star\hspace{0.1cm}
        \star\hspace{0.1cm}
        \star\hspace{0.1cm}
        \star\hspace{0.1cm}
        \star\hspace{0.1cm}
        \star
    \]
    
    \[|\hspace{0.67cm}|\]
\end{frame}

\begin{frame}[t]{Stars and Bars I}
    \begin{definition}
    The number of ways to distribute $m$ things to $n$ people can be represented rearranging $m$ stars and $n-1$ bars. The number of ways to do this is
    \[
    \binom{m+n-1}{n-1}
    \]
    \end{definition}
\end{frame}

\begin{frame}[t]{Quick Quiz}
    How many ways are there for nonnegative integers $a,b,c,d$ to satisfy \[a+b+c+d=18\]  
    
    \vspace{2cm}
    
    For example, the quadruplet $(a,b,c,d)=(6,7,4,1)$ works.
\end{frame}

\begin{frame}[t]{What if...?}
    ...you needed the integers in the previous slide to be positive?

\end{frame}

\begin{frame}[t]{What if...?}
    ...you needed the integers in the previous slide to be positive?

    \vspace{2cm}

    Use $a=a'+1,\;b=b'+1,\;c=c'+1,\;d=d'+1$ (this is because $a,b,c,d$ are positive so we allow $a',b',c',d'$ to be nonnegative).
    So 
    \[
        a'+b'+c'+d'=14
    \]
\end{frame}

\begin{frame}[t]{Stars and Bars II}
    Using this same logic from above we can also apply this to other numbers. If we needed at least 2, we can proceed to substitute the $a=a'+2$.
    \vspace{2cm}

    For example if we have 3 people and 8 objects, person A needs 1 object, person B needs 2 objects and person C needs 3 objects. 
    $A+B+C=8$ , $A=A'+1$, $B=B'+2$ and $C=C'+3$
    Thus we get $A'+B'+C'=2$ and we will have 2 bars and 2 stars with no restrictions.

\end{frame}

\begin{frame}[t]{Quicker Quiz}
    Marty, a magical magician has 67 Twix up his sleeve for 10 kids. If each kid shall receive at least 5 Twix, how many ways are Marty to give the kids the Twix?
    \vspace{2cm}
    
    Varun lost his interest in chess and is now a full time mathematician. He had a total of 30 indistinguishable boards and would like to donate them to 5 organizations named Archimedes, Bernoulli, Cayley, Dirac, and Euler. If Varun would like to give Archimedes 1 board, Bernoulli 2 boards... , how many ways can he distribute them?
\end{frame}

\section{Worksheet!}

\end{document}